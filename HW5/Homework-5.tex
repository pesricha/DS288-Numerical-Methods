%%%%%%%%%%%%%%%%%%%%%%%%% NOTE %%%%%%%%%%%%%%%%%%%%%%%%%%%%
%% You can ignore everything from here until             %%
%% "Question 1: Introduction"                            %%
%%%%%%%%%%%%%%%%%%%%%%%%%%%%%%%%%%%%%%%%%%%%%%%%%%%%%%%%%%%
\documentclass[8pt]{article}
\usepackage{amsmath, amsfonts, amsthm, amssymb}  % Some math symbols
\usepackage{fullpage}
\usepackage{graphicx}
\usepackage[x11names, rgb]{xcolor}
\usepackage{graphicx}
\usepackage{tikz}
\usepackage{tcolorbox}
\usetikzlibrary{decorations,arrows,shapes}
\usepackage{float} % Add this package to control float placement
\usepackage{etoolbox}
\usepackage{enumerate}
\usepackage{listings}
\lstset{
    language=Python,           % Set the language of the code
    basicstyle=\footnotesize\ttfamily,
    keywordstyle=\color{blue}, % Set color for keywords
    commentstyle=\color{gray}, % Set color for comments
    stringstyle=\color{red},   % Set color for strings
    numbers=left,              % Display line numbers on the left
    numberstyle=\tiny\color{gray}, % Style for line numbers
    frame=single,              % Add a frame around the code
    breaklines=true            % Allow line breaking
}


\setlength{\parindent}{0pt}
\setlength{\parskip}{5pt plus 1pt}

\newcommand{\N}{\mathbb N}
\newcommand{\E}{\mathbb E}
\newcommand{\V}{Var}
\renewcommand{\P}{\mathbb P}
\newcommand{\f}{\frac}


\newcommand{\nopagenumbers}{
    \pagestyle{empty}
}

\def\indented#1{\list{}{}\item[]}
\let\indented=\endlist

\providetoggle{questionnumbers}
\settoggle{questionnumbers}{true}
\newcommand{\noquestionnumbers}{
    \settoggle{questionnumbers}{false}
}

\newcounter{questionCounter}
\newenvironment{question}[2][\arabic{questionCounter}]{%
    \addtocounter{questionCounter}{1}%
    \setcounter{partCounter}{0}%
    \vspace{.25in} \hrule \vspace{0.4em}%
        \noindent{\bf \iftoggle{questionnumbers}{#1: }{}#2}%
    \vspace{0.8em} \hrule \vspace{.10in}%
}{$ $\newpage}

\newcounter{partCounter}[questionCounter]
\renewenvironment{part}[1][\alph{partCounter}]{%
    \addtocounter{partCounter}{1}%
    \vspace{.10in}%
    \begin{indented}%
       {\bf (#1)} %
}{\end{indented}}

\def\show#1{\ifdefempty{#1}{}{#1\\}}

\newcommand{\header}{%
\begin{center}
    {\Large \show\myhwname}
    \show\myname
    \show\myemail
    \show\mysection
    \show\hwname
\end{center}}

\usepackage{hyperref} % for hyperlinks
\hypersetup{
    colorlinks=true,
    linkcolor=blue,
    filecolor=magenta,      
    urlcolor=blue,
}

%%%%%%%%%%%%%%%%% Identifying Information %%%%%%%%%%%%%%%%%
%% For 312, we'd rather you DIDN'T tell us who you are   %%
%% in your homework so that we're not biased when        %%
%% So, even if you fill this information in, it will not %%
%% show up in the document unless you uncomment \header  %%
%% below                                                 %%
%%%%%%%%%%%%%%%%%%%%%%%%%%%%%%%%%%%%%%%%%%%%%%%%%%%%%%%%%%%
\newcommand{\myhwname}{DS288 (AUG) 3:0 Numerical Methods }
\newcommand{\myname}{Naman Pesricha }
\newcommand{\myemail}{namanp@iisc.ac.in}
\newcommand{\hwname}{\textbf{Homework-4}}
\newcommand{\mysection}{SR - 24115}
%%%%%%%%%%%%%%%%%%%%%%%%%%%%%%%%%%%%%%%%%%%%%%%%%%%%%%%%%%%

%%%%%%%%%%%%%%%%%%% Document Options %%%%%%%%%%%%%%%%%%%%%%
\noquestionnumbers
\nopagenumbers
%%%%%%%%%%%%%%%%%%%%%%%%%%%%%%%%%%%%%%%%%%%%%%%%%%%%%%%%%%%

\begin{document}
\header

\begin{question}{Q1 Derive Simpsons Rule with error term by using

 $$\int_{x_0}^{x_2} f(x) \, dx = a_0 f(x_0) + a_1 f(x_1) + a_2 f(x_2) + k f^{(4)}(\xi)$$
 Find $a_0$, $a_1$, and $a_2$ from the fact that Simpson's rule is exact for $f(x) = x^n$ when
 n = 0, 1, 2, and 3. Then find $k$ by applying the integration formula to f(x) = $x^4$. [3
    points]
}
We use equispaced points for Simpson's rule $x_1 = x_{0} + h $ and $x_2 = x_{0} + 2h $.  We can substitute these values to simplify our system of equations.
We have the following 4 equations
    
    \begin{equation}
        \int_{x_0}^{x_2} x^0 \, dx = a_{0} + a_{1} + a_{2} =  x_2 -x_0  = 2 h 
    \end{equation}

    \begin{equation}
        \int_{x_0}^{x_2} x^1 \, dx = a_{0} x_{0} + a_{1} \left(h + x_{0}\right) + a_{2} \left(2 h + x_{0}\right) = - \frac{x_{0}^{2}}{2} + \frac{\left(2 h + x_{0}\right)^{2}}{2}
    \end{equation}

    \begin{equation}
        \int_{x_0}^{x_2} x^2 \, dx =a_{0} x_{0}^{2} + a_{1} \left(h + x_{0}\right)^{2} + a_{2} \left(2 h + x_{0}\right)^{2} = - \frac{x_{0}^{3}}{3} + \frac{\left(2 h + x_{0}\right)^{3}}{3}
    \end{equation}

    \begin{equation}
        \int_{x_0}^{x_2} x^3 \, dx = a_{0} x_{0}^{3} + a_{1} \left(h + x_{0}\right)^{3} + a_{2} \left(2 h + x_{0}\right)^{3} = - \frac{x_{0}^{4}}{4} + \frac{\left(2 h + x_{0}\right)^{4}}{4}
    \end{equation}

    Using equation 2 , 3 and 4 we get our system of equations :

$$
    X = \left[\begin{matrix}
    x_{0} & h + x_{0} & 2 h + x_{0} \\
    x_{0}^{2} & \left(h + x_{0}\right)^{2} & \left(2 h + x_{0}\right)^{2} \\
    x_{0}^{3} & \left(h + x_{0}\right)^{3} & \left(2 h + x_{0}\right)^{3}
    \end{matrix}\right]     
    b = \left[\begin{matrix}
    - \frac{x_{0}^{2}}{2} + \frac{\left(2 h + x_{0}\right)^{2}}{2} \\
    - \frac{x_{0}^{3}}{3} + \frac{\left(2 h + x_{0}\right)^{3}}{3} \\
    - \frac{x_{0}^{4}}{4} + \frac{\left(2 h + x_{0}\right)^{4}}{4}
    \end{matrix}\right]
    a = \begin{bmatrix} a_0 \\ a_1 \\ a_2 \end{bmatrix} 
$$

   $$ Xa = b \implies a = X^{-1}b $$

$$
a = \begin{bmatrix} a_0 \\ a_1 \\ a_2 \end{bmatrix} 
=  
\left[\begin{matrix}
    x_{0} & h + x_{0} & 2 h + x_{0} \\
    x_{0}^{2} & \left(h + x_{0}\right)^{2} & \left(2 h + x_{0}\right)^{2} \\
    x_{0}^{3} & \left(h + x_{0}\right)^{3} & \left(2 h + x_{0}\right)^{3}
    \end{matrix}\right]^{-1}
\left[\begin{matrix}
    - \frac{x_{0}^{2}}{2} + \frac{\left(2 h + x_{0}\right)^{2}}{2} \\
    - \frac{x_{0}^{3}}{3} + \frac{\left(2 h + x_{0}\right)^{3}}{3} \\
    - \frac{x_{0}^{4}}{4} + \frac{\left(2 h + x_{0}\right)^{4}}{4}
    \end{matrix}\right]
$$

\begin{tcolorbox}
    $$
        a = \begin{bmatrix} a_0 \\ a_1 \\ a_2 \end{bmatrix} 
        =  
        \left[\begin{matrix}
        \frac{2 h^{2} + 3 h x_{0} + x_{0}^{2}}{2 h^{2} x_{0}} & \frac{- 3 h - 2 x_{0}}{2 h^{2} x_{0}} & \frac{1}{2 h^{2} x_{0}} \\
        \frac{- 2 h x_{0} - x_{0}^{2}}{h^{3} + h^{2} x_{0}} & \frac{2}{h^{2}} & - \frac{1}{h^{3} + h^{2} x_{0}} \\
        \frac{h x_{0} + x_{0}^{2}}{4 h^{3} + 2 h^{2} x_{0}} & \frac{- h - 2 x_{0}}{4 h^{3} + 2 h^{2} x_{0}} & \frac{1}{4 h^{3} + 2 h^{2} x_{0}}
        \end{matrix}\right]
        \left[\begin{matrix}
            - \frac{x_{0}^{2}}{2} + \frac{\left(2 h + x_{0}\right)^{2}}{2} \\
            - \frac{x_{0}^{3}}{3} + \frac{\left(2 h + x_{0}\right)^{3}}{3} \\
            - \frac{x_{0}^{4}}{4} + \frac{\left(2 h + x_{0}\right)^{4}}{4}
            \end{matrix}\right]= 
            \left[\begin{matrix} 
        \frac{h}{3} \\
        \frac{4 h}{3} \\
        \frac{h}{3}
        \end{matrix}\right]
    $$
\end{tcolorbox}

\begin{center}
\begin{tcolorbox}[width=0.5\textwidth]
$$a_0 = \frac{h}{3},\ a_1 = \frac{4h}{3},\ \text{and}\ a_2 = \frac{h}{3}$$
\end{tcolorbox}
\end{center}

These are computed using eq 2, 3 and 4 and also satisfies equation 1 :  $a_0 + a_1 + a_2 = \frac{h}{3}+\frac{4h}{3}+\frac{h}{3}  = 2h$. Hence it satisfies all 4 equations.


\hfill \\
For f(x) = $x^4$, we get \fbox{$f^{4}(\xi) = 4*3*2*1*x^0 = 24$}, therefore we get the following equation:

\begin{equation}
    \int_{x_0}^{x_2} x^4 \, dx = a_{0} x_{0}^{4} + a_{1} \left(h + x_{0}\right)^{4} + a_{2} \left(2 h + x_{0}\right)^{4} + 24k= - \frac{x_{0}^{5}}{5} + \frac{\left(2 h + x_{0}\right)^{5}}{5}
\end{equation}

$$
\frac{h}{3} x_{0}^{4} + \frac{4h}{3} \left(h + x_{0}\right)^{4} + \frac{h}{3} \left(2 h + x_{0}\right)^{4} + 24k= - \frac{x_{0}^{5}}{5} + \frac{\left(2 h + x_{0}\right)^{5}}{5} 
$$

$$
 24k= - \frac{x_{0}^{5}}{5} + \frac{\left(2 h + x_{0}\right)^{5}}{5} - \frac{h}{3} x_{0}^{4}  - \frac{4h}{3} \left(h + x_{0}\right)^{4} - \frac{h}{3} \left(2 h + x_{0}\right)^{4} 
$$

\centering Simplifying LHS will yield
$$
24k = -\frac{x_{0}^{5}}{5} + \frac{32 h^{5}}{5} + 16 h^{4} x_{0} + 16 h^{3} x_{0}^{2} + 8 h^{2} x_{0}^{3} + 2 h x_{0}^{4} + \frac{x_{0}^{5}}{5}  - \left[\frac{h x_{0}^{4}}{3}\right] - \left[\frac{4 h^{5}}{3} + \frac{16 h^{4} x_{0}}{3} + 8 h^{3} x_{0}^{2} + \frac{16 h^{2} x_{0}^{3}}{3} + \frac{4 h x_{0}^{4}}{3}\right] -
$$
$$
\left[\frac{16 h^{5}}{3} + \frac{32 h^{4} x_{0}}{3} + 8 h^{3} x_{0}^{2} + \frac{8 h^{2} x_{0}^{3}}{3} + \frac{h x_{0}^{4}}{3}\right]
$$

\centering All terms will get cancelled except the coefficient of $h^5$.
$$
24k = \frac{32 h^{5}}{5} - \frac{16 h^{5}}{3} - \frac{4 h^{5}}{3} = \frac{-4}{15} h^4
$$

\begin{tcolorbox}[width=0.3\textwidth]
$$
k = \frac{-h^5}{90}
$$ 
\end{tcolorbox}


\begin{tcolorbox}
    Therefore the final Simpson's rule with error term is:
    \large $$\int_{x_0}^{x_2} f(x) \, dx = \frac{h}{3} f(x_0) + \frac{4h}{3} f(x_1) + \frac{h}{3} f(x_2) - \frac{h^5}{90}f^{(4)}(\xi)$$
\end{tcolorbox}

\end{question}
\begin{question}{Q2 Apply Romberg Integration to the following integrals until \( R_{n-1, n-1} \) and \( R_{n, n} \) agree to within \( 10^{-5} \). Report the value of \( n \) and the number of function evaluations. Also, compute the result to that obtained from the Trapezoidal rule for the same number \( n \) (note that you already calculate this value to get \( R_{n, n} \)). \textbf{[3.5 points]}

\begin{center}
    \begin{tabular}{ccc}
        (a) & \( \int_{0}^{1} x^{1/3} \, dx \); & (b) \( \int_{0}^{1} x^2 e^{-x} \, dx \)
    \end{tabular}
\end{center}}

\end{question}

\begin{question}{Q3 Approximate the integrals in Problem 2(a) and 2(b) using Gaussian Quadrature with
 n=2, 3, 4, and 5. Report the number of function evaluations and compare the results
 with those obtained using Romberg Integration in Problem-2. [3.5 points]}
\end{question}
    
\end{document}



