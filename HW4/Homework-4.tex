%%%%%%%%%%%%%%%%%%%%%%%%% NOTE %%%%%%%%%%%%%%%%%%%%%%%%%%%%
%% You can ignore everything from here until             %%
%% "Question 1: Introduction"                            %%
%%%%%%%%%%%%%%%%%%%%%%%%%%%%%%%%%%%%%%%%%%%%%%%%%%%%%%%%%%%
\documentclass[8pt]{article}
\usepackage{amsmath, amsfonts, amsthm, amssymb}  % Some math symbols
\usepackage{fullpage}
\usepackage{graphicx}
\usepackage[x11names, rgb]{xcolor}
\usepackage{graphicx}
\usepackage{tikz}
\usepackage{tcolorbox}
\usetikzlibrary{decorations,arrows,shapes}
\usepackage{float} % Add this package to control float placement
\usepackage{etoolbox}
\usepackage{enumerate}
\usepackage{listings}
\lstset{
    language=Python,           % Set the language of the code
    basicstyle=\footnotesize\ttfamily,
    keywordstyle=\color{blue}, % Set color for keywords
    commentstyle=\color{gray}, % Set color for comments
    stringstyle=\color{red},   % Set color for strings
    numbers=left,              % Display line numbers on the left
    numberstyle=\tiny\color{gray}, % Style for line numbers
    frame=single,              % Add a frame around the code
    breaklines=true            % Allow line breaking
}


\setlength{\parindent}{0pt}
\setlength{\parskip}{5pt plus 1pt}

\newcommand{\N}{\mathbb N}
\newcommand{\E}{\mathbb E}
\newcommand{\V}{Var}
\renewcommand{\P}{\mathbb P}
\newcommand{\f}{\frac}


\newcommand{\nopagenumbers}{
    \pagestyle{empty}
}

\def\indented#1{\list{}{}\item[]}
\let\indented=\endlist

\providetoggle{questionnumbers}
\settoggle{questionnumbers}{true}
\newcommand{\noquestionnumbers}{
    \settoggle{questionnumbers}{false}
}

\newcounter{questionCounter}
\newenvironment{question}[2][\arabic{questionCounter}]{%
    \addtocounter{questionCounter}{1}%
    \setcounter{partCounter}{0}%
    \vspace{.25in} \hrule \vspace{0.4em}%
        \noindent{\bf \iftoggle{questionnumbers}{#1: }{}#2}%
    \vspace{0.8em} \hrule \vspace{.10in}%
}{$ $\newpage}

\newcounter{partCounter}[questionCounter]
\renewenvironment{part}[1][\alph{partCounter}]{%
    \addtocounter{partCounter}{1}%
    \vspace{.10in}%
    \begin{indented}%
       {\bf (#1)} %
}{\end{indented}}

\def\show#1{\ifdefempty{#1}{}{#1\\}}

\newcommand{\header}{%
\begin{center}
    {\Large \show\myhwname}
    \show\myname
    \show\myemail
    \show\mysection
    \show\hwname
\end{center}}

\usepackage{hyperref} % for hyperlinks
\hypersetup{
    colorlinks=true,
    linkcolor=blue,
    filecolor=magenta,      
    urlcolor=blue,
}

%%%%%%%%%%%%%%%%% Identifying Information %%%%%%%%%%%%%%%%%
%% For 312, we'd rather you DIDN'T tell us who you are   %%
%% in your homework so that we're not biased when        %%
%% So, even if you fill this information in, it will not %%
%% show up in the document unless you uncomment \header  %%
%% below                                                 %%
%%%%%%%%%%%%%%%%%%%%%%%%%%%%%%%%%%%%%%%%%%%%%%%%%%%%%%%%%%%
\newcommand{\myhwname}{DS288 (AUG) 3:0 Numerical Methods }
\newcommand{\myname}{Naman Pesricha }
\newcommand{\myemail}{namanp@iisc.ac.in}
\newcommand{\hwname}{\textbf{Homework-4}}
\newcommand{\mysection}{SR - 24115}
%%%%%%%%%%%%%%%%%%%%%%%%%%%%%%%%%%%%%%%%%%%%%%%%%%%%%%%%%%%

%%%%%%%%%%%%%%%%%%% Document Options %%%%%%%%%%%%%%%%%%%%%%
\noquestionnumbers
\nopagenumbers
%%%%%%%%%%%%%%%%%%%%%%%%%%%%%%%%%%%%%%%%%%%%%%%%%%%%%%%%%%%

\begin{document}
\header

\begin{question}{Q1
Increment $\theta$ from $0^o$
to $360^o$
in steps of $1^o$ and compute $\phi$ and $d\phi/d\theta$ at each point.
Report plot of $\phi$ and $d\phi/d\theta$ versus $\theta$. For the first derivative, compute both a first
forward difference and a centered difference approximation. Plot the two curves on the
same graph. How do these two curves compare?. Which do you expect to be more
accurate? When using the Newton Algorithm from Homework–2, as you increment $\theta$
use the previously found solution as an initial starting guess for the next value of $\theta$. [4
points]
}

\begin{figure}[H]
    \centering
    \begin{minipage}{0.45\textwidth}
        \centering
        \includegraphics[width=\textwidth]{HW4/imagesmall/phi vs theta.png}
        \caption{Caption for the first image}
        \label{fig:image1}
    \end{minipage}\hfill
    \begin{minipage}{0.45\textwidth}
        \centering
        \includegraphics[width=\textwidth]{HW4/imagesmall/dphi vs theta.png}
        \caption{Caption for the second image}
        \label{fig:image2}
    \end{minipage}
    \caption{Overall caption for both images}
\end{figure}

\end{question}

\begin{question} {Q2 Now solve the second linkage problem by determining \(\alpha\) from your computed values of \(\phi\) and using Newton’s Method on the second linkage system to compute \(\beta\), \(\frac{d\beta}{dt}\) (i.e., the angular velocity in rad/sec), and \(\frac{d^2\beta}{dt^2}\) (i.e., the angular acceleration in rad/sec\(^2\)).

Make plots of these quantities as a function of \(\theta\), and in the case of the derivatives, compute and plot both forward and centered approximations as before. Note that

\[
\frac{d\beta}{dt} = \omega \frac{d\beta}{d\theta}; \quad \frac{d^2\beta}{dt^2} = \omega^2 \frac{d^2\beta}{d\theta^2}
\]

where \(\omega\) is the rotating speed of the driving gear (in this case, assume that \(\omega = 450 \ \text{rad/min}\)).

As a check on your answers, you should find that when \(\theta = 100^\circ\), the angular velocity is near \(10 \ \text{rad/sec}\) and the angular acceleration is close to \(-25 \ \text{rad/sec}^2\). Be careful with units while programming!
}
    
\end{question}

\end{document}



